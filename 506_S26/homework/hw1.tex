\documentclass[12pt]{article}
\usepackage{fullpage, amsmath, amssymb, amsthm, amscd, setspace, bm, graphicx, indentfirst, multirow, tikz, enumerate, verbatim, appendix}
\usepackage{adjustbox,amsfonts,array,graphicx,booktabs,tabularx,multirow,multicol,stmaryrd, tabu}
\usepackage[utf8]{inputenc}
\usepackage{cite}
\usepackage{hyperref}
\usetikzlibrary{arrows.meta}
\title{Math 506, Spring 2026 -- Homework 1}
\date{}
\setlength{\parskip}{0.5cm}
\setlength{\parindent}{0cm}

\newtheorem{theorem}{Theorem}[section]
\newtheorem{definition}[theorem]{Definition}
\newtheorem{lemma}[theorem]{Lemma}
\newtheorem{proposition}[theorem]{Proposition}
\newtheorem{corollary}[theorem]{Corollary}
\newtheorem{remark}[theorem]{Remark}
\newtheorem{example}[theorem]{Example}

\newcommand{\Z}{\mathbb{Z}}
\newcommand{\R}{\mathbb{R}}
\newcommand{\Q}{\mathbb{Q}}
\newcommand{\C}{\mathbb{C}}
\newcommand{\N}{\mathbb{N}}
\renewcommand{\t}[1]{\text{#1}}
\newcommand{\ba}{\overline}
\newcommand{\Hom}{\text{Hom}}
\newcommand{\End}{\text{End}}
\newcommand{\Aut}{\text{Aut}}
\newcommand{\Mat}{\text{Mat}}
\newcommand{\Sym}{\text{Sym}}
\newcommand{\wt}{\text{wt}}

\makeatletter
\newcommand{\Wedge}{\@ifnextchar^\@Wedge{\@Wedge^{\,}}}
\def\@Wedge^#1{\mathop{\bigwedge\nolimits^{\!#1}}}
\makeatother

\begin{document} \maketitle
\vspace{-80pt}

\textbf{Due:} Wednesday, February 11th, at 9:00am via Gradescope.

\textbf{Instructions:} Students should complete and submit all problems. All assertions require proof unless otherwise stated. Typesetting your homework using LaTeX is recommended.

For this homework, unless otherwise stated all groups are finite and all representations are finite dimensional and complex.

\begin{enumerate}
\item[1.] Let $V$ and $W$ be $G$-representations.

\begin{enumerate}
\item In Lecture 2, we defined $G$-actions on duals and tensor products; combining these gives $V^*\otimes W$ the structure of a $G$-representation. Under the (vector space) identification
\[
V^*\otimes W \cong \Hom(V,W) \qquad\qquad (\varphi\otimes w)(v) := \varphi(w)v,
\]
we obtain a $G$-action on $\Hom(V,W)$. Show that this is the same representation that we defined on $\Hom(V,W)$ in Lecture 2.

\item Show that $\phi\in\Hom(V,W)$ lies in the isotypic component of the trivial representation if and only if it is $G$-equivariant.
\end{enumerate}

\item[2.] The \emph{permutation representation} $V_{\t{perm}}$ of $S_3$ is its permutation action on $\C^3$:
\[
w\cdot e_i := e_{w(i)}, \qquad i=1,2,3.
\]
Without using character theory, show that $V_{\t{perm}} \cong V_{\t{triv}} \oplus V_{\t{ref}}$. That is, find a one-dimensional fixed subspace (i.e. a copy of the trivial representation), and a complementary two-dimensional subspace that is $S_3$-invariant and irreducible (and therefore is the reflection representation, which you can assume is the only two-dimensional irreducible representation of $S_3$.)

\item[3.] We generalize the setting of the previous problem. Let $X$ be a finite set with a $G$-action. The \emph{permutation representation} associated to this action is the vector space $V = V_X = \{v_x | x\in X\}$, and the action $gv_x := v_{g\cdot x}$ turns $V$ into a representation.
\begin{enumerate}
    \item Show that $\chi_V(g)$ is the number of elements of $X$ fixed by $g$.
    \item Show that the number of (distinct) $G$-orbits in $X$ equals the multiplicity of the trivial representation inside $V$.
\end{enumerate}

\item[4.] Let $G$ be a finite group. Prove that the following representations are isomorphic:
\[
R_1 = \langle v_g | g\in G\rangle, \qquad g\cdot v_h := v_{gh};
\]
\[
R_2 = \langle v_g | g\in G\rangle, \qquad g\cdot v_h := v_{hg^{-1}};
\]
\[
R_3 = \langle f:G\to\C\rangle, \qquad g\cdot f(h) := f(g^{-1}h);
\]
\[
R_4 = \langle f:G\to\C\rangle, \qquad g\cdot f(h) := f(hg).
\]

(Any of these is called the \emph{regular representation} of $G$. $R_1$ is the permutation representation associated to the left action of $G$ on itself, while $R_2$ is the permutation representation associated to the right action of $G$ on itself. $R_1$ and $R_3$ are each called the \emph{left regular representation}, while $R_2$ and $R_4$ are called the \emph{right regular representation}. The definitions in terms of functions tend to be more useful when working with infinite groups.)

\item[5.] Let $V$ be an irreducible representation of the finite group $G$. To prove Maschke's Theorem, we constructed a $G$-invariant Hermitian inner product. Show that this inner product is unique up to scalar multiple. (\emph{Hint: Hermitian inner products on $\C^n$ are classified by $n\times n$ positive-definite Hermitian matrices.})

\item[6.] Show that for any representation $V$
\[
\chi_{\Sym^2 V} = \frac{1}{2}[\chi_V(g)^2 + \chi_V(g^2)]
\qquad\t{and}\qquad
\chi_{\Wedge^2 V} = \frac{1}{2}[\chi_V(g)^2 - \chi_V(g^2)].
\]
Use this to show that
\[
V\otimes V\cong \Sym^2 V\oplus \Wedge^2 V.
\]


\item[7.] Let $V := V_{\t{ref}}$ be the reflection representation of $S_3$ Use the character table of $S_3$
\begin{center}
    \renewcommand{\arraystretch}{1.5}
    \begin{tabular}{|c|c|c|c|} \hline
         & $1$ & $3$ & $2$\\
         $S_3$ & $()$ & $(12)$ & $(123)$\\ \hline
         $\chi_{\t{triv}}$ & 1 & 1 & 1\\ \hline
         $\chi_{\t{sgn}}$ & 1 & -1 & 1\\ \hline
         $\chi_{\t{ref}}$ & 2 & 0 & -1\\ \hline
    \end{tabular}
\end{center}

to determine the decomposition of $V^{\otimes n}, n\ge 1$, into irreducibles.




\end{enumerate}

\end{document}