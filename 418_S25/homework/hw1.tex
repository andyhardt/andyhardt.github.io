\documentclass[12pt]{article}
\usepackage{fullpage, amsmath, amssymb, amsthm, amscd, setspace, bm, graphicx, indentfirst, multirow, tikz, enumerate, verbatim, appendix}

\usepackage{adjustbox,amsfonts,array,graphicx,booktabs,tabularx,multirow,multicol,stmaryrd, tabu}
\usepackage[utf8]{inputenc}
\usepackage{cite}
\usepackage{hyperref}
\usetikzlibrary{arrows.meta}
\title{Math 418, Spring 2025 -- Homework 1}
\date{}
\setlength{\parskip}{0.5cm}
\setlength{\parindent}{0cm}

\newtheorem{theorem}{Theorem}[section]
\newtheorem{definition}[theorem]{Definition}
\newtheorem{lemma}[theorem]{Lemma}
\newtheorem{proposition}[theorem]{Proposition}
\newtheorem{corollary}[theorem]{Corollary}
\newtheorem{remark}[theorem]{Remark}
\newtheorem{example}[theorem]{Example}

\newcommand{\Z}{\mathbb{Z}}
\newcommand{\R}{\mathbb{R}}
\newcommand{\Q}{\mathbb{Q}}
\newcommand{\C}{\mathbb{C}}
\newcommand{\F}{\mathbb{F}}
\newcommand{\N}{\mathbb{N}}
\newcommand{\U}{\mathcal{U}}
\newcommand{\B}{\mathcal{B}}
\newcommand{\T}{\mathbb{T}}
\newcommand{\real}{\textrm{Re }}
\newcommand{\imag}{\textrm{Im }}
\newcommand{\deriv}[2]{\frac{d #1}{d #2}}
\newcommand{\sig}[1]{\sum_{#1 =1}^\infty}
\newcommand{\un}[1]{\bigcup_{#1 =1}^\infty}
\newcommand{\inter}[1]{\bigcap_{#1 =1}^\infty}
\newcommand{\cyc}[2]{\genfrac{[}{]}{0pt}{}{#1}{#2}}
\newcommand{\px}[2]{\genfrac{\{}{\}}{0pt}{}{#1}{#2}}
\newcommand{\wh}[1]{\widehat{#1}}
\newcommand{\del}{\partial}
\newcommand{\st}{ \; \big | \:}
\newcommand{\ba}{\overline}
\newcommand{\Int}{\mathrm{Int}}
\newcommand{\Cl}{\mathrm{Cl}}
\newcommand{\rel}{\mathrm{rel}}
\newcommand{\Hom}{\text{Hom}}
\newcommand{\End}{\text{End}}
\newcommand{\Mor}{\text{Mor}}
\newcommand{\Ind}{\text{Ind}}
\newcommand{\Irr}{\text{Irr}}
\newcommand{\vect}[1]{\boldsymbol{#1}}
\newcommand{\mfk}[1]{\mathfrak{#1}}
\newcommand{\wt}{\text{wt}}
\newcommand{\vx}[1]{\mathtt{#1}}
%\renewcommand{\wt}[1]{\mathtt{#1}}
\begin{document} \maketitle
\vspace{-80pt}

\textbf{Due:} Wednesday, January 29th, at 9:00am via Gradescope.

\textbf{Instructions:} Students should complete and submit all problems. Textbook problems are from Dummit and Foote, \emph{Abstract Algebra, 3rd Edition}. All assertions require proof, unless otherwise stated. Typesetting your homework using LaTeX is recommended, and will gain you 1 bonus point per assignment.

\begin{enumerate}

\item[1.] \textbf{Dummit and Foote \#7.1.3:} \textit{Let $R$ be a ring with identity and let $S$ be a subring of $R$ containing the identity. Prove that if $u$ is a unit in $S$ then $u$ is a unit in $R$. Show by example that the converse is false.}

\item[2.] \textbf{Dummit and Foote \#7.1.11:} \textit{Prove that if $R$ is an integral domain and $x^2 = 1$ for some $x\in R$ then $x = \pm 1$.}

\item[3.] \textbf{Dummit and Foote \#7.2.1:} \textit{Let $p(x) = 2x^3 - 3x^2 + 4x - 5$ and let $q(x) = 7x^3 + 33x - 4$. In each of parts (a), (b) and (c) compute $p(x) + q (x)$ and $p(x)q(x)$ under the assumption that the coefficients of the two given polynomials are taken from the specified ring (where the integer coefficients are taken mod $n$ in parts (b) and (c)).}

\begin{enumerate}
    \item $R=\Z$.

    \item $R = \Z/2\Z$.
    
    \item $R = \Z/3\Z$.
    
\end{enumerate}

\item[4.] \textbf{Dummit and Foote \#7.3.2:} \textit{Prove that the rings $\Z[x]$ and $\Q[x]$ are not isomorphic.}

\item[5.] \textbf{Dummit and Foote \#7.4.15:} \textit{Let $x^2 + x + 1$ be an element of the polynomial ring $E = \F_2[x]$ and use the bar notation to denote passage to the quotient ring $\F_2[x]/(x^2+x+1)$.}

\begin{enumerate}
    \item Prove that $E$ has 4 elements: $\ba{0}, \ba{1}, \ba{x}$, and $\ba{x+1}$.
    
    \item Write out the $4\times 4$ addition table for $E$ and deduce that the additive group $E$ is isomorphic to the Klein 4-group.

    \item Write out the $4\times 4$ multiplication table for $E$ and prove that $E^\times$ is isomorphic to the cyclic group of order 3. Deduce that $E$ is a field.
\end{enumerate}

\item[6.] \textit{Consider $R = \Z[\sqrt{-5}]$ with the (non-Euclidean) norm $N:R\to \Z_{\ge 0}$ given by $N(a) = |a|^2$. Note that $N(a\cdot b) = N(a)N(b)$.}

\begin{enumerate}
    \item Prove that $a\in R$ is a unit if and only if $N(a) = 1$. Find all the units in $R$.
    
    \item Recall that $r\in R$ is irreducible if whenever $r = ab$ then one of $a$ or $b$ is a unit. Use the norm to show that 2, 3, $1+\sqrt{-5}$, and $1-\sqrt{-5}$ are all irreducible elements of $R$
    
    \item Show that $2, 3, 1 + \sqrt{-5}$, and $1 - \sqrt{-5}$ are not unit multiples of one another, proving that $R$ lacks unique factorization since $6 = 2\cdot 3 = (1+\sqrt{-5})(1-\sqrt{-5})$.
\end{enumerate}

 
\item[7.] \textit{Let $R$ be an integral domain. Recall that $g$ is a greatest common divisor of two elements $a, b\in R$
if $g$ divides $a$ and $b$, and if $d$ divides $a$ and $b$ then $d$ divides $g$.}

\begin{enumerate}
    \item Show that if $g$ and $g'$ are two gcds of $a,b\in R$, $g' = ug$ for some unit $u$.
    
    \item Let $R = \Z[\sqrt{-5}]$. Prove that $6$ and $2+2\sqrt{-5}$ have no gcd. \emph{(Hint: Use the fact that $2$ and $1+\sqrt{-5}$ are both common divisors of these elements)}
\end{enumerate}

\end{enumerate}


\end{document}
