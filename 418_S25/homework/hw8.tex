\documentclass[12pt]{article}
\usepackage{fullpage, amsmath, amssymb, amsthm, amscd, setspace, bm, graphicx, indentfirst, multirow, tikz, enumerate, verbatim, appendix}

\usepackage{adjustbox,amsfonts,array,graphicx,booktabs,tabularx,multirow,multicol,stmaryrd, tabu}
\usepackage[utf8]{inputenc}
\usepackage{cite}
\usepackage{hyperref}
\usetikzlibrary{arrows.meta}
\title{Math 418, Spring 2025 -- Homework 8}
\date{}
\setlength{\parskip}{0.5cm}
\setlength{\parindent}{0cm}

\newtheorem{theorem}{Theorem}[section]
\newtheorem{definition}[theorem]{Definition}
\newtheorem{lemma}[theorem]{Lemma}
\newtheorem{proposition}[theorem]{Proposition}
\newtheorem{corollary}[theorem]{Corollary}
\newtheorem{remark}[theorem]{Remark}
\newtheorem{example}[theorem]{Example}

\newcommand{\Z}{\mathbb{Z}}
\newcommand{\R}{\mathbb{R}}
\newcommand{\Q}{\mathbb{Q}}
\newcommand{\C}{\mathbb{C}}
\newcommand{\F}{\mathbb{F}}
\newcommand{\N}{\mathbb{N}}
\newcommand{\U}{\mathcal{U}}
\newcommand{\B}{\mathcal{B}}
\newcommand{\T}{\mathbb{T}}
\newcommand{\real}{\textrm{Re }}
\newcommand{\imag}{\textrm{Im }}
\newcommand{\deriv}[2]{\frac{d #1}{d #2}}
\newcommand{\sig}[1]{\sum_{#1 =1}^\infty}
\newcommand{\un}[1]{\bigcup_{#1 =1}^\infty}
\newcommand{\inter}[1]{\bigcap_{#1 =1}^\infty}
\newcommand{\cyc}[2]{\genfrac{[}{]}{0pt}{}{#1}{#2}}
\newcommand{\px}[2]{\genfrac{\{}{\}}{0pt}{}{#1}{#2}}
\newcommand{\wh}[1]{\widehat{#1}}
\newcommand{\del}{\partial}
\newcommand{\st}{ \; \big | \:}
\newcommand{\ba}{\overline}
\newcommand{\Int}{\mathrm{Int}}
\newcommand{\Cl}{\mathrm{Cl}}
\newcommand{\rel}{\mathrm{rel}}
\newcommand{\Hom}{\text{Hom}}
\newcommand{\End}{\text{End}}
\newcommand{\Mat}{\text{Mat}}
\newcommand{\Aut}{\text{Aut}}
\newcommand{\Mor}{\text{Mor}}
\newcommand{\Ind}{\text{Ind}}
\newcommand{\Irr}{\text{Irr}}
\newcommand{\Gal}{\text{Gal}}
\newcommand{\Fix}{\text{Fix}}
\renewcommand{\char}{\text{char }}
\newcommand{\vect}[1]{\boldsymbol{#1}}
\newcommand{\mfk}[1]{\mathfrak{#1}}
\newcommand{\wt}{\text{wt}}
\newcommand{\vx}[1]{\mathtt{#1}}
%\renewcommand{\wt}[1]{\mathtt{#1}}
\begin{document} \maketitle
\vspace{-80pt}

\textbf{Due:} Wednesday, April 9th, at 9:00am via Gradescope.

\textbf{Instructions:} Students should complete and submit all problems. Textbook problems are from Dummit and Foote, \emph{Abstract Algebra, 3rd Edition}. All assertions require proof, unless otherwise stated. Typesetting your homework using LaTeX is recommended, and will gain you 1 bonus point per assignment.

\begin{enumerate}

\item[1.] \textbf{Dummit and Foote \#14.2.6:} \textit{Let $K = \Q(\sqrt[8]{2}, i)$ and let $F_1 = \Q(i), F_2 = \Q(\sqrt{2}) , F_3 = \Q(\sqrt{-2})$. Prove that $\Gal(K/F_1) = \Z/8\Z, \Gal(K/F_2) = D_8, \Gal(K/F_3) = Q_8$. (Hint: use the example in Section 14.2, and the diagrams on pages 580-1)}

\item[2.] \textbf{Dummit and Foote \#14.2.7:} \textit{Determine all the subfields of the splitting field of $x^8 - 2$ which are Galois over $\Q$. (Hint: use the example in Section 14.2, and the diagrams on pages 580-1)}

\item[3.] \textbf{Dummit and Foote \#14.2.14:} \textit{Show that $\Q(\sqrt{2+\sqrt{2}})$ is a cyclic quartic field, i.e., is a Galois extension of degree 4 with
cyclic Galois group.}

\item[4.] \textit{Let $K/F$ be a Galois extension of degree $n$ with $G = \Gal(K/F)$. For $\alpha\in K$, define the norm and trace of $\alpha$ by \[N_{K/F}(\alpha) := \prod_{\sigma\in G} \sigma(\alpha), \qquad \text{and} \qquad Tr_{K/F}(\alpha) = \sum_{\sigma\in G} \sigma(\alpha).\] Let $m_{\alpha,F}(x) = x^d + a_{d-1}x^{d-1} + \cdots + a_1x + a_0$.}

\begin{enumerate}
    \item Show that $N_{K/F}(\alpha) = (-1)^n a_0^{n/d}$ and $Tr_{K/F}(\alpha) = -\frac{n}{d}a_{d-1}$.

    \item Show that \[N_{K/F}(\alpha\beta) = N_{K/F}(\alpha)N_{K/F}(\beta)\qquad \text{and} \qquad Tr_{K/F}(\alpha+\beta) = Tr_{K/F}(\alpha)+Tr_{K/F}(\beta).\]

    \item Show that $N_{K/F}(a\alpha) = a^n N_{K/F}(\alpha)$ and $Tr_{K/F}(a\alpha) = aTr_{K/F}(\alpha)$ for all $a\in F$, In particular show that $N_{K/F}(a) = a^n$ and $Tr_{K/F}(a) = na$ for all $a\in F$.

\end{enumerate}

\item[5.] \textbf{Dummit and Foote \#14.5.3:} \textit{Determine the quadratic equation satisfied by the period $\alpha = \zeta_5 + \zeta_5^{-1}$ of the 5th root of unity $\zeta_5$. Determine the quadratic equation satisfied by $\zeta_5$ over $\Q(\alpha)$ and use this to explicitly solve for the 5th root of unity.}

\item[6.] \textbf{Dummit and Foote \#14.5.7:} \textit{Show that complex conjugation restricts to the automorphism $\sigma_{-1}\in \Gal(\Q(\zeta_n)/\Q)$ of the cyclotomic field of nth roots of unity. Show that the field $K^+ = \Q(\zeta_n + \zeta_n^{-1})$ is the
subfield of real elements in K = $\Q(\zeta_n)$, called the maximal real subfield of K.}

\item[7.] \textbf{Dummit and Foote \#14.5.10:} \textit{Prove that $\Q(\sqrt[3]{2})$ is not a subfield of any cyclotomic field over $\Q$.}

\end{enumerate}

\end{document}
