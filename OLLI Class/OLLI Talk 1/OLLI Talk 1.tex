\documentclass{beamer}
\usetheme{Boadilla}
\newcommand{\ba}{\overline}
\definecolor{resulthead}{RGB}{205,205,235}
\definecolor{termshead}{RGB}{242,218,195}
\begin{document}

\title[Math and Proofs]{Math and Proofs Class 1}

\begin{frame}[plain]
\titlepage
\end{frame}

\begin{frame}{What do we need to create a mathematical framework?}
\begin{itemize}
\item Undefined Terms
\item Assumptions
\item Method of Proof
\end{itemize}
\end{frame}

\begin{frame}{Euclid's Elements (for geometry)}
Undefined Terms
\begin{enumerate}
\item Point
\item Line
\item Plane
\end{enumerate}
\end{frame}

\begin{frame}{Euclid's Elements (cont.)}
Common Notions
\begin{enumerate}
\item That which are equal to the same thing are also equal to one another
\item If equals are added to equals, the wholes are equal
\item If equals are subtracted from equals, the remainders are equal
\item Things which coincide with one another are equal to one another
\item The whole is greater than the part
\end{enumerate}
\end{frame}

\begin{frame}{Euclid's Elements (cont.)}
Postulates
\begin{enumerate}
\item It is possible to draw a straight line from any point to any other point
\item It is possible to extend a line segment continuously in both directions
\item It is possible to describe a circle with any center and any radius
\item It is true that all right angles are equal to one another
\item It is true that, if a straight line falling on two straight lines makes the interior angles on the same side less than two right angles, the two straight lines, if produced indefinitely, intersect on that side.
\end{enumerate}
\end{frame}

\begin{frame}{Things to Note}
\begin{itemize}
\item Not very precise
\item More assumptions needed
\item Common notions and postulates are both just assumptions, or axioms
\item Parallel postulate was important historically
\end{itemize}
\end{frame}

\begin{frame}{A couple of Elements Proofs}
Proposition 1.4: If two triangles have two sides equal to two sides, respectively, and have the angles enclosed by the straight lines equal, then they will also have the base equal to the base, and the triangle will be equal to the triangle, and the remaining angles subtended by the equal sides will be equal to the corresponding remaining angles.
\end{frame}

\begin{frame}{A couple of Elements Proofs (cont.)}
Proposition 1.6: If a triangle has two angles equal to one another, then the sides subtending the equal angles will also be equal to one another.
\end{frame}

\begin{frame}{Peano Axioms (for arithmetic)}
\begin{enumerate}
\item Zero is a number
\item If $a$ is a number, the successor of $a$ is a number
\item Zero is not the successor of a number
\item Two numbers of which the successors are equal are themselves equal
\item If a set $S$ of numbers contains zero and also the successor of every number in $S$, then every number is in $S$.
\end{enumerate}
\end{frame}

\begin{frame}{Let's use the Peano Axioms to do some (sort of boring, but essential) things}
\begin{itemize}
\item Define addition
\item Show that addition is commutative
\item Show that addition is associative
\item Define multiplication
\item Show that multiplication is commutative, associative, and distributive over addition
\end{itemize}
\end{frame}

\begin{frame}{Next Time}
\begin{itemize}
\item Start to look at set theory, which we can use to describe \textit{both} geometry and arithmetic, and also everything else in math!
\item Maybe do a couple proofs in groups or on your own (interest?)
\end{itemize}
\end{frame}

\end{document}