\documentclass[12pt]{article}
\usepackage{fullpage, amsmath, amssymb, amsthm, amscd, setspace, bm, graphicx, indentfirst, multirow, tikz, enumerate, verbatim, appendix}

\usepackage{adjustbox,amsfonts,array,graphicx,booktabs,tabularx,multirow,multicol,stmaryrd, tabu}
\usepackage[utf8]{inputenc}
\usepackage{cite}
\usepackage{hyperref}
\usetikzlibrary{arrows.meta}
\title{Math 418, Spring 2024 -- Homework 2}
\date{}
\setlength{\parskip}{0.5cm}
\setlength{\parindent}{0cm}

\newtheorem{theorem}{Theorem}[section]
\newtheorem{definition}[theorem]{Definition}
\newtheorem{lemma}[theorem]{Lemma}
\newtheorem{proposition}[theorem]{Proposition}
\newtheorem{corollary}[theorem]{Corollary}
\newtheorem{remark}[theorem]{Remark}
\newtheorem{example}[theorem]{Example}

\newcommand{\Z}{\mathbb{Z}}
\newcommand{\R}{\mathbb{R}}
\newcommand{\Q}{\mathbb{Q}}
\newcommand{\C}{\mathbb{C}}
\newcommand{\F}{\mathbb{F}}
\newcommand{\N}{\mathbb{N}}
\newcommand{\U}{\mathcal{U}}
\newcommand{\B}{\mathcal{B}}
\newcommand{\T}{\mathbb{T}}
\newcommand{\real}{\textrm{Re }}
\newcommand{\imag}{\textrm{Im }}
\newcommand{\deriv}[2]{\frac{d #1}{d #2}}
\newcommand{\sig}[1]{\sum_{#1 =1}^\infty}
\newcommand{\un}[1]{\bigcup_{#1 =1}^\infty}
\newcommand{\inter}[1]{\bigcap_{#1 =1}^\infty}
\newcommand{\cyc}[2]{\genfrac{[}{]}{0pt}{}{#1}{#2}}
\newcommand{\px}[2]{\genfrac{\{}{\}}{0pt}{}{#1}{#2}}
\newcommand{\wh}[1]{\widehat{#1}}
\newcommand{\del}{\partial}
\newcommand{\st}{ \; \big | \:}
\newcommand{\ba}{\overline}
\newcommand{\Int}{\mathrm{Int}}
\newcommand{\Cl}{\mathrm{Cl}}
\newcommand{\rel}{\mathrm{rel}}
\newcommand{\Hom}{\text{Hom}}
\newcommand{\End}{\text{End}}
\newcommand{\Mor}{\text{Mor}}
\newcommand{\Ind}{\text{Ind}}
\newcommand{\Irr}{\text{Irr}}
\newcommand{\vect}[1]{\boldsymbol{#1}}
\newcommand{\mfk}[1]{\mathfrak{#1}}
\newcommand{\wt}{\text{wt}}
\newcommand{\vx}[1]{\mathtt{#1}}
%\renewcommand{\wt}[1]{\mathtt{#1}}
\begin{document} \maketitle
\vspace{-80pt}

\textbf{Due:} Wednesday, January 31st, at 9:00am via Gradescope.

\textbf{Instructions:} Students should complete and submit all problems. Textbook problems are from Dummit and Foote, \emph{Abstract Algebra, 3rd Edition}. All assertions require proof, unless otherwise stated. Typesetting your homework using LaTeX is recommended, and will gain you 2 bonus points per assignment.

\begin{enumerate}

\item[1.] Let $R$ be a Principal Ideal Domain, and $I$ an ideal of $R$. Prove that every ideal of $S := R/I$
is principal. ($S$ may fail to be an integral domain, and hence is not always a P.I.D itself; for example, $R = \Z$ and $I = 4\Z$.)

\item[2.] Dummit and Foote \#8.2.5: Let $R$ be the quadratic integer ring $\Z[\sqrt{-5}]$. Define the ideals $I_2 = (2, 1 + \sqrt{-5}), I_3 = (3, 2 + \sqrt{-5})$, and $I_3' = (3, 2 - \sqrt{-5})$.

\begin{enumerate}
    \item Prove that $I_2, I_3$, and $I_3'$ are nonprincipal ideals in R. \emph{(Hint: use Homework 1 Problem 6)}

    \item Prove that the product of two nonprincipal ideals can be principal by showing that $I_2^2 = (2)$.

    \item Prove similarly that $I_2I_3 = (1-\sqrt{-5})$ and $I_2I_3' = (1+\sqrt{-5})$ are principal. Conclude that the principal ideal $(6)$ is the product of 4 ideals: $(6) = I_2^2I_3I_3'$.

    
\end{enumerate}

\item[3.] Dummit and Foote \#8.2.7: An integral domain $R$ in which every ideal generated by two elements is principal (i.e., for every $a, b\in R, (a, b) = (d)$ for some $d\in R$) is called a Bezout Domain.

\begin{enumerate}
    \item Prove that the integral domain $R$ is a Bezout Domain if and only if every pair of elements $a, b$ of $R$ has a g.c.d. $d$ in $R$ that can be written as an $R$-linear combination of $a$ and $b$, i.e., $d = ax + by$ for some $x, y\in R$.

    \item Prove that every finitely generated ideal of a Bezout Domain is principal.

    \item Let $F$ be the fraction field of the Bezout Domain $R$ \emph{(since $R$ is an integral domain, this has the form $F = \{a/b | a\in R, b\in R\setminus\{0\}\}$, with $a/b = c/d$ if and only if $ad = bc$.)}. Prove that every element of $F$ can be written in the form $a/b$ with $a, b\in R$ and $a$ and $b$ relatively prime (1 is a gcd of $a$ and $b$).

\end{enumerate}

\item[4.] Dummit and Foote \#8.3.6: 

\begin{enumerate}
    \item Prove that the quotient ring $\Z[i]/(1 + i)$ is a field of order 2.
    
    \item Let $q\in \Z$ be a prime with $q\equiv 3 \mod 4$. Prove that the quotient ring $\Z[i]/(q)$ is a field with $q^2$ elements.
    
    \item Let $p\in\Z$ be a prime with $p\equiv 1 \mod 4$ and write $p = \pi\ba{\pi}$  as in Proposition 18 ($\ba{\pi}$ is the complex conjugate of $\pi$). Show that the hypotheses for the Chinese Remainder Theorem (Theorem 17 in Section 7 .6) are satisfied and that $\Z[i]/(p)\cong \Z[i]/(\pi)\times \Z[i]/(\ba{\pi})$ as rings. Show that the quotient ring $\Z[i]/(p)$ has order $p^2$ and conclude that $\Z[i]/(\pi)$ and $\Z[i]/(\ba{\pi})$ are both fields of order $p$.

\end{enumerate}

\item[5.] Dummit and Foote \#8.3.11: Prove that $R$ is a P.I.D. if and only if $R$ is a U.F.D. that is
also a Bezout Domain.


\item[6.] Dummit and Foote \#9.3.1: Let $R$ be an integral domain with quotient field $F$ and let $p(x)$ be a monic polynomial in $R[x]$. Assume that $p(x) = a(x)b(x)$ where $a (x)$ and $b(x)$ are monic polynomials in $F[x]$ of smaller degree than $p(x)$. Prove that if $a(x)\notin R[x]$ then $R$ is not a Unique Factorization Domain. Deduce that $\Z[2\sqrt{2}]$ is not a U.F.D.


\end{enumerate}


\end{document}
