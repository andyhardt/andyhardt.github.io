\documentclass[12pt]{article}
\usepackage{fullpage, amsmath, amssymb, amsthm, amscd, setspace, bm, graphicx, indentfirst, multirow, tikz, enumerate, verbatim, appendix}

\usepackage{adjustbox,amsfonts,array,graphicx,booktabs,tabularx,multirow,multicol,stmaryrd, tabu}
\usepackage[utf8]{inputenc}
\usepackage{cite}
\usepackage{hyperref}
\usetikzlibrary{arrows.meta}
\title{Math 418, Spring 2024 -- Homework 7}
\date{}
\setlength{\parskip}{0.5cm}
\setlength{\parindent}{0cm}

\newtheorem{theorem}{Theorem}[section]
\newtheorem{definition}[theorem]{Definition}
\newtheorem{lemma}[theorem]{Lemma}
\newtheorem{proposition}[theorem]{Proposition}
\newtheorem{corollary}[theorem]{Corollary}
\newtheorem{remark}[theorem]{Remark}
\newtheorem{example}[theorem]{Example}

\newcommand{\Z}{\mathbb{Z}}
\newcommand{\R}{\mathbb{R}}
\newcommand{\Q}{\mathbb{Q}}
\newcommand{\C}{\mathbb{C}}
\newcommand{\F}{\mathbb{F}}
\newcommand{\N}{\mathbb{N}}
\newcommand{\U}{\mathcal{U}}
\newcommand{\B}{\mathcal{B}}
\newcommand{\T}{\mathbb{T}}
\newcommand{\real}{\textrm{Re }}
\newcommand{\imag}{\textrm{Im }}
\newcommand{\deriv}[2]{\frac{d #1}{d #2}}
\newcommand{\sig}[1]{\sum_{#1 =1}^\infty}
\newcommand{\un}[1]{\bigcup_{#1 =1}^\infty}
\newcommand{\inter}[1]{\bigcap_{#1 =1}^\infty}
\newcommand{\cyc}[2]{\genfrac{[}{]}{0pt}{}{#1}{#2}}
\newcommand{\px}[2]{\genfrac{\{}{\}}{0pt}{}{#1}{#2}}
\newcommand{\wh}[1]{\widehat{#1}}
\newcommand{\del}{\partial}
\newcommand{\st}{ \; \big | \:}
\newcommand{\ba}{\overline}
\newcommand{\Int}{\mathrm{Int}}
\newcommand{\Cl}{\mathrm{Cl}}
\newcommand{\rel}{\mathrm{rel}}
\newcommand{\Hom}{\text{Hom}}
\newcommand{\End}{\text{End}}
\newcommand{\Mat}{\text{Mat}}
\newcommand{\Aut}{\text{Aut}}
\newcommand{\Mor}{\text{Mor}}
\newcommand{\Ind}{\text{Ind}}
\newcommand{\Irr}{\text{Irr}}
\newcommand{\Gal}{\text{Gal}}
\renewcommand{\char}{\text{char }}
\newcommand{\vect}[1]{\boldsymbol{#1}}
\newcommand{\mfk}[1]{\mathfrak{#1}}
\newcommand{\wt}{\text{wt}}
\newcommand{\vx}[1]{\mathtt{#1}}
%\renewcommand{\wt}[1]{\mathtt{#1}}
\begin{document} \maketitle
\vspace{-80pt}

\textbf{Due:} \textbf{Friday}, March 29th, at 9:00am via Gradescope.

\textbf{Instructions:} Students should complete and submit all problems. Textbook problems are from Dummit and Foote, \emph{Abstract Algebra, 3rd Edition}. All assertions require proof, unless otherwise stated. Typesetting your homework using LaTeX is recommended, and will gain you 2 bonus points per assignment.

\begin{enumerate}


\item[1.] \textbf{Dummit and Foote \#13.3.3:} \textit{Prove that an algebraically closed field must be infinite.}

\item[2.] \textbf{Dummit and Foote \#13.3.4:} \textit{Construct the finite field of 16 elements and find a generator for the multiplicative group. How many generators are there?}

\item[3.] \textbf{Dummit and Foote \#13.3.8:} \textit{Determine the splitting field of the polynomial $f(x) = x^p - x - a$ over $\F_p$ where $a\ne 0, a\in F_p$. Show explicitly that the Galois group is cyclic.}

\item[4.] \textbf{Dummit and Foote \#13.4.2:} \textit{Find a primitive element for $\Q(\sqrt{2}, \sqrt{3}, \sqrt{5})$ over $\Q$.}

\item[5.] \textbf{Dummit and Foote \#13.4.3:} \textit{Let $F$ be a field contained in the ring $\Mat_n(\Q)$ of $n \times n$ matrices over $\Q$. Here, $\Q\subseteq \Mat_n(\Q)$ is identified with the scalar diagonal matrices by the inclusion \[q\mapsto qI = \begin{bmatrix} q &&& \\ &q&& \\ &&\cdots& \\ &&&q\end{bmatrix}.\] Prove that $[F : \Q] \le n$.} (\emph{I do have a hint for this one, if you ask})

\end{enumerate}

\end{document}
