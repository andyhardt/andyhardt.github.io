\documentclass[12pt]{article}
\usepackage{fullpage, amsmath, amssymb, amsthm, amscd, setspace, bm, graphicx, indentfirst, multirow, tikz, enumerate, verbatim, appendix}

\usepackage{adjustbox,amsfonts,array,graphicx,booktabs,tabularx,multirow,multicol,stmaryrd, tabu}
\usepackage[utf8]{inputenc}
\usepackage{cite}
\usepackage{hyperref}
\usetikzlibrary{arrows.meta}
\title{Math 418, Spring 2024 -- Homework 9}
\date{}
\setlength{\parskip}{0.5cm}
\setlength{\parindent}{0cm}

\newtheorem{theorem}{Theorem}[section]
\newtheorem{definition}[theorem]{Definition}
\newtheorem{lemma}[theorem]{Lemma}
\newtheorem{proposition}[theorem]{Proposition}
\newtheorem{corollary}[theorem]{Corollary}
\newtheorem{remark}[theorem]{Remark}
\newtheorem{example}[theorem]{Example}

\newcommand{\Z}{\mathbb{Z}}
\newcommand{\R}{\mathbb{R}}
\newcommand{\Q}{\mathbb{Q}}
\newcommand{\C}{\mathbb{C}}
\newcommand{\F}{\mathbb{F}}
\newcommand{\N}{\mathbb{N}}
\newcommand{\U}{\mathcal{U}}
\newcommand{\B}{\mathcal{B}}
\newcommand{\T}{\mathbb{T}}
\newcommand{\real}{\textrm{Re }}
\newcommand{\imag}{\textrm{Im }}
\newcommand{\deriv}[2]{\frac{d #1}{d #2}}
\newcommand{\sig}[1]{\sum_{#1 =1}^\infty}
\newcommand{\un}[1]{\bigcup_{#1 =1}^\infty}
\newcommand{\inter}[1]{\bigcap_{#1 =1}^\infty}
\newcommand{\cyc}[2]{\genfrac{[}{]}{0pt}{}{#1}{#2}}
\newcommand{\px}[2]{\genfrac{\{}{\}}{0pt}{}{#1}{#2}}
\newcommand{\wh}[1]{\widehat{#1}}
\newcommand{\del}{\partial}
\newcommand{\st}{ \; \big | \:}
\newcommand{\ba}{\overline}
\newcommand{\Int}{\mathrm{Int}}
\newcommand{\Cl}{\mathrm{Cl}}
\newcommand{\rel}{\mathrm{rel}}
\newcommand{\Hom}{\text{Hom}}
\newcommand{\End}{\text{End}}
\newcommand{\Mat}{\text{Mat}}
\newcommand{\Aut}{\text{Aut}}
\newcommand{\Mor}{\text{Mor}}
\newcommand{\Ind}{\text{Ind}}
\newcommand{\Irr}{\text{Irr}}
\newcommand{\Gal}{\text{Gal}}
\newcommand{\Fix}{\text{Fix}}
\renewcommand{\char}{\text{char }}
\newcommand{\vect}[1]{\boldsymbol{#1}}
\newcommand{\mfk}[1]{\mathfrak{#1}}
\newcommand{\wt}{\text{wt}}
\newcommand{\vx}[1]{\mathtt{#1}}
%\renewcommand{\wt}[1]{\mathtt{#1}}
\begin{document} \maketitle
\vspace{-80pt}

\textbf{Due:} Wednesday, April 24th, at 9:00am via Gradescope.

\textbf{Instructions:} Students should complete and submit all problems. Textbook problems are from Dummit and Foote, \emph{Abstract Algebra, 3rd Edition}. All assertions require proof, unless otherwise stated. Typesetting your homework using LaTeX is recommended, and will gain you 2 bonus points per assignment.

\begin{enumerate}

\item[1.] \textbf{Dummit and Foote \#14.6.2a} \textit{Determine the Galois group of the polynomial $f(x) = x^3 - x^2 - 4$}

\item[2.] \textbf{Dummit and Foote \#14.6.10} \textit{Determine the Galois group of $x^5 + x - 1$.} (\emph{Hint: see D \& F Proposition 14.21}

\item[3.] \textit{Let $p_k(x_1,\ldots,x_n) = x_1^k + x_2^k + \cdots + x_n^k$ be the \emph{power sum symmetric function}, and let $e_k(x_1,\ldots,x_n) = \sum_{i_1<\ldots < i_k} x_{i_1}\cdots x_{i_k}$ be the elementary symmetric function. Let \[E(t) = \sum_{r=0}^\infty e_r(x_1,\ldots,x_n) t^r, \hspace{20pt} P(t) = \sum_{r=1}^\infty p_r(x_1,\ldots,x_n) t^{r-1}.\] Prove that \[E(t) = \prod_{i=1}^n (1+x_it), \hspace{20pt} P(t) = \sum_{i=1}^n \frac{x_i}{1-x_it} = \sum_{i=1}^n \frac{d}{dt} \ln\frac{1}{1-x_it}.\]}

\item[4.] \textbf{Dummit and Foote \#14.6.22} \textit{Let $f(x)$ be a monic polynomial of degree n with roots $\alpha_1,\ldots,\alpha_n$. Let $e_i$ be the elementary symmetric function of degree $i$ in the roots and define $e_i = 0$ for $i > n$. Let $p_i = \alpha_1^i + \cdots + \alpha_n^i, i\ge 0$, be the sum of the $i$th powers of the roots of $f(x)$ Prove Newton's formulas: \[p_n - e_1p_{n-1} + e_2p_{n-2} + \cdots + (-1)^{n-1}e_{n-1}p_1 + (-1)^n n e_n = 0.\]} \emph{(Hint: use solution to previous problem)}

\item[5.] \textbf{Dummit and Foote \#14.7.1} \textit{Use Cardano's Formulas to solve the equation $f(x) = x^3 + x^2 - 2 = 0$. In particular show that the equation has the real root \[\frac{1}{3}\left(\sqrt[3]{26 + 15\sqrt{3}} + \sqrt[3]{26 - 15\sqrt{3}} - 1\right).\] Show directly that the roots of this cubic are $1 , -1\pm i$. Explain this by proving that \[\sqrt[3]{26 + 15\sqrt{3}} = 2+\sqrt{3}, \hspace{20pt} \sqrt[3]{26 - 15\sqrt{3}} = 2-\sqrt{3}\] so that \[\sqrt[3]{26 + 15\sqrt{3}} + \sqrt[3]{26 - 15\sqrt{3}} = 4.\]}

\item[6.] \textbf{Dummit and Foote \#14.7.17} \textit{Let $D\in \Z$ be a squarefree integer and let $a\in \Q$ be a nonzero rational number. Show that $\Q(\sqrt{a\sqrt{D}})$ cannot be a cyclic extension of degree 4 over $\Q$ (i.e. $\Gal(\Q(\sqrt{a\sqrt{D}})/\Q)$ cannot be $\Z/4\Z$).}

\end{enumerate}

\end{document}
