\documentclass[12pt]{article}
\usepackage{fullpage, amsmath, amssymb, amsthm, amscd, setspace, bm, graphicx, indentfirst, multirow, tikz, enumerate, verbatim, appendix}

\usepackage{adjustbox,amsfonts,array,graphicx,booktabs,tabularx,multirow,multicol,stmaryrd, tabu}
\usepackage[utf8]{inputenc}
\usepackage{cite}
\usepackage{hyperref}
\usetikzlibrary{arrows.meta}
\title{Math 418, Spring 2024 -- Homework 4}
\date{}
\setlength{\parskip}{0.5cm}
\setlength{\parindent}{0cm}

\newtheorem{theorem}{Theorem}[section]
\newtheorem{definition}[theorem]{Definition}
\newtheorem{lemma}[theorem]{Lemma}
\newtheorem{proposition}[theorem]{Proposition}
\newtheorem{corollary}[theorem]{Corollary}
\newtheorem{remark}[theorem]{Remark}
\newtheorem{example}[theorem]{Example}

\newcommand{\Z}{\mathbb{Z}}
\newcommand{\R}{\mathbb{R}}
\newcommand{\Q}{\mathbb{Q}}
\newcommand{\C}{\mathbb{C}}
\newcommand{\F}{\mathbb{F}}
\newcommand{\N}{\mathbb{N}}
\newcommand{\U}{\mathcal{U}}
\newcommand{\B}{\mathcal{B}}
\newcommand{\T}{\mathbb{T}}
\newcommand{\real}{\textrm{Re }}
\newcommand{\imag}{\textrm{Im }}
\newcommand{\deriv}[2]{\frac{d #1}{d #2}}
\newcommand{\sig}[1]{\sum_{#1 =1}^\infty}
\newcommand{\un}[1]{\bigcup_{#1 =1}^\infty}
\newcommand{\inter}[1]{\bigcap_{#1 =1}^\infty}
\newcommand{\cyc}[2]{\genfrac{[}{]}{0pt}{}{#1}{#2}}
\newcommand{\px}[2]{\genfrac{\{}{\}}{0pt}{}{#1}{#2}}
\newcommand{\wh}[1]{\widehat{#1}}
\newcommand{\del}{\partial}
\newcommand{\st}{ \; \big | \:}
\newcommand{\ba}{\overline}
\newcommand{\Int}{\mathrm{Int}}
\newcommand{\Cl}{\mathrm{Cl}}
\newcommand{\rel}{\mathrm{rel}}
\newcommand{\Hom}{\text{Hom}}
\newcommand{\End}{\text{End}}
\newcommand{\Mor}{\text{Mor}}
\newcommand{\Ind}{\text{Ind}}
\newcommand{\Irr}{\text{Irr}}
\newcommand{\vect}[1]{\boldsymbol{#1}}
\newcommand{\mfk}[1]{\mathfrak{#1}}
\newcommand{\wt}{\text{wt}}
\newcommand{\vx}[1]{\mathtt{#1}}
%\renewcommand{\wt}[1]{\mathtt{#1}}
\begin{document} \maketitle
\vspace{-80pt}

\textbf{Due:} Wednesday, February 14th, at 9:00am via Gradescope.

\textbf{Instructions:} Students should complete and submit all problems. Textbook problems are from Dummit and Foote, \emph{Abstract Algebra, 3rd Edition}. All assertions require proof, unless otherwise stated. Typesetting your homework using LaTeX is recommended, and will gain you 2 bonus points per assignment.

\begin{enumerate}

\item[1.] \textbf{Dummit and Foote \#13.2.1:} \textit{Let $\F$ be a finite field of characteristic $p$. Prove that $|\F| = p^n$ for some positive integer $n$.}

\item[2.] \textbf{Dummit and Foote \#13.2.4:} \textit{Determine the degree over $\Q$ of $2 + \sqrt{3}$ and of $1 + \sqrt[3]{2} + \sqrt[3]{4}$.}

\item[3.] \textbf{Dummit and Foote \#13.2.5:} \textit{Let $F = \Q(i)$. Prove that $x^3 - 2$ and $x^3 - 3$ are irreducible over $F$.}

\item[4.] \textbf{Dummit and Foote \#13.2.7:} \textit{Prove that $\Q(\sqrt{2} + \sqrt{3}) = \Q(\sqrt{2}, \sqrt{3})$. Conclude that $[\Q(\sqrt{2} + \sqrt{3}) : \Q] = 4$. Find an irreducible polynomial satisfied by $\sqrt{2} + \sqrt{3}$.}

\vspace{10pt}
\centering \textcolor{blue}{MORE PROBLEMS TO COME!!!}

\end{enumerate}


\end{document}
