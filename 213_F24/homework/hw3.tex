\documentclass{article}
%%%%%%%%%%%%%
% Loads packages
%%%%%%%%%%%%%
\usepackage[table]{xcolor}
\usepackage[utf8]{inputenc}
\usepackage[colorlinks=true,linkcolor=blue]{hyperref}
\usepackage{geometry} %package needed to set margins
\usepackage{fancyhdr}
\usepackage{graphicx}
\usepackage{amsmath}
\usepackage{amsthm}
\usepackage{mdframed}
\usepackage{tikz}
\usepackage{amsfonts}

\usepackage{listings}% http://ctan.org/pkg/listings
\lstset{
  basicstyle=\ttfamily,
  mathescape
}


\pagestyle{fancy}
\fancyhf{}
\chead{\textbf{Homework 3}}
\lhead{Math 213, Fall 2024}
\rhead{Due Sunday, 9/15 at 11:59pm}

%%%%%%%%%%%%%
% Sets margins
%%%%%%%%%%%%%
\newgeometry{left=1.5in,right=1in,top=1in,bottom=1in}
\setlength\headsep{3pt}

%%%%%%%%%%%%%
% Creates problem and solution environments
%%%%%%%%%%%%%

% Solution Environment
\newenvironment{solution}{\begin{proof}[Solution]}{\end{proof}}

% Problem Environment
\newenvironment{problem}[1]
    {\begin{mdframed}[default]
    \textbf{Problem #1:}
    }
    {\end{mdframed}
    }
    
%%%%%%%%%%%
% Custom Commands
%%%%%%%%%%%
\newcommand{\gOne}{\cellcolor{green!50!white} 1}
\newcommand{\rZero}{\cellcolor{red!50!white} 0}

\begin{document}

\begin{problem}{\S 3.2: 2(a,b,e,f)}
Determine whether each of these functions is $O(x^2)$:
\begin{enumerate}
    \item[(a)] $f(x) = 17x + 11$.
    \item[(b)] $f(x) = x^2 + 1000$.
    \item[(e)] $f(x) = 2^x$.
    \item[(f)] $f(x) = \lfloor x \rfloor \cdot \lceil x \rceil$.
\end{enumerate}
\end{problem}

\begin{problem}{\S 3.2: 8}
Find the least integer $n$ such that $f(x)$ is $O(x^n)$ for each of these functions.
\begin{enumerate}
    \item[(a)] $f(x) = 2x^2 + x^3\log{x}$.
    \item[(b)] $f(x) = 3x^5 + (\log{x})^4$.
    \item[(c)] $f(x) = (x^4+x^2+1)/(x^4+1)$.
    \item[(d)] $f(x) = (x^3+5\log{x})/(x^4+1)$.
\end{enumerate}
\end{problem}

\begin{problem}{\S 3.2: 17}
Suppose that $f(x)$, $g(x)$, and $h(x)$ are functions such that $f(x)$ is $O(g(x))$ and $g(x)$ is $O(h(x))$. Show that $f(x)$ is $O(h(x))$.
\end{problem}

\begin{problem}{\S 3.2: 26}
Give a big-$O$ estimate for each of these functions. For the function $g$ in your estimate $f(x)$ is $O(g(x))$, use a simple function $g$ of the smallest order.
\begin{enumerate}
    \item[(a)] $f(x) = (n^3+n^2\log{n})(\log{n}+1) + (17\log{n} + 19)(n^3+2)$.
    \item[(b)] $f(x) = (2^n + n^2)(n^3 + 3^n)$.
    \item[(c)] $f(x) = (n^n + n2^n + 5^n)(n!+5^n)$.
\end{enumerate}
\end{problem}

\begin{problem}{\S 3.2: 28(a,b,c,d)}
Determine whether each of the following functions is $\Omega(x)$ and whether it is $\Theta(x)$.
\begin{enumerate}
    \item[(a)] $f(x) = 10$.
    \item[(b)] $f(x) = 3x+7$.
    \item[(c)] $f(x) = x^2 + x + 1$.
    \item[(d)] $f(x) = 5\log{x}$.
\end{enumerate}
\end{problem}

\begin{problem}{Extra}
Explain what it means for a function to be
\begin{enumerate}
    \item[(a)] $O(1)$.
    \item[(b)] $\Omega(1)$.
    \item[(c)] $\Theta(1)$.
\end{enumerate}
\end{problem}

\begin{problem}{\S 5.1: 4}
Let $P(n)$ be the statement that $1^3 + 2^3 + \cdots + n^3 = (n(n+1)/2)^2$ for the positive integer $n$.
\begin{enumerate}
    \item[(a)] What is the statement $P(1)$?
    \item[(b)] Show that $P(1)$ is true, completing the basis step of the proof.
    \item[(c)] What is the inductive hypothesis?
    \item[(d)] What do you need to prove in the inductive step?
    \item[(e)] Complete the inductive step, identifying where you use the inductive hypothesis.
    \item[(f)] Explain why these steps show that this formula is true whenever $n$ is a positive integer.
\end{enumerate}
\end{problem}

\begin{problem}{\S 5.1: 6}
Prove that $1 \cdot 1! + 2 \cdot 2! + \cdots + n \cdot n! = (n+1)!-1$ whenever $n$ is a positive integer.
\end{problem}

\begin{problem}{\S 5.1: 8}
Prove that $2 - 2 \cdot 7 + 2 \cdot 7^2 - \cdots + 2(-7)^n = (1-(-7)^{n+1})/4$ whenever $n$ is a nonnegative integer.
\end{problem}

\begin{problem}{\S 5.1: 20}
Prove that $3^n < n!$ if $n$ is an integer greater than $6$.
\end{problem}

\begin{problem}{\S 5.1: 34}
Prove that $6$ divides $n^3 - n$ whenever $n$ is a nonnegative integer.
\end{problem}

\begin{problem}{\S 5.1: 49}
What is wrong with this ``proof'' that all horses are the same color?

\vspace{3mm}
\noindent Let $P(n)$ be the proposition that all the horses in a set of $n$ horses are the same color.

\vspace{2mm}
\noindent \emph{Basis Step:} Clearly, $P(1)$ is true.

\vspace{2mm}
\noindent \emph{Inductive Step:} Assume that $P(k)$ is true, so that all the horses in any set of $k$ horses are the same color. Consider any $k+1$ horses: number these horses as $1, 2, 3, \dots, k, k+1$. Now the first $k$ of these horses all must have the same color. Because the set of the first $k$ horses and the set of the last $k$ horses overlap, all $k+1$ must be the same color. This shows that $P(k+1)$ is true and finishes the proof by induction.
\end{problem}

\begin{problem}{\S 5.1: 51}
What is wrong with this ``proof''?

\vspace{3mm}
\noindent ``\emph{Theorem}'': For every positive integer $n$, if $x$ and $y$ are positive integers with $\textrm{max}(x,y) = n$, then $x = y$.

\vspace{2mm}
\noindent \emph{Basis Step:} Suppose that $n=1$. If $\textrm{max}(x,y)=1$ and $x$ and $y$ are positive integers, we have $x = 1$ and $y = 1$.

\vspace{2mm}
\noindent \emph{Inductive Step:} Let $k$ be a positive integer. Assume that whenever $\textrm{max}(x,y) = k$ and $x$ and $y$ are positive integers, then $x = y$. Now let $\textrm{max}(x,y) = k+1$, where $x$ and $y$ are positive integers. Then $\textrm{max}(x-1,y-1) = k$, so by the inductive hypothesis $x-1 = y-1$. It follows that $x = y$, completing the inductive step.
\end{problem}


\end{document}