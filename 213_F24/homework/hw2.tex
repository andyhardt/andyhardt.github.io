\documentclass{article}
%%%%%%%%%%%%%
% Loads packages
%%%%%%%%%%%%%
\usepackage[table]{xcolor}
\usepackage[utf8]{inputenc}
\usepackage[colorlinks=true,linkcolor=blue]{hyperref}
\usepackage{geometry} %package needed to set margins
\usepackage{fancyhdr}
\usepackage{graphicx}
\usepackage{amsmath}
\usepackage{amsthm}
\usepackage{mdframed}
\usepackage{tikz}
\usepackage{amsfonts}

\usepackage{listings}% http://ctan.org/pkg/listings
\lstset{
  basicstyle=\ttfamily,
  mathescape
}


\pagestyle{fancy}
\fancyhf{}
\chead{\textbf{Homework 2}}
\lhead{Math 213, Fall 2024}
\rhead{Due Sunday, 9/8 at 11:59pm}

%%%%%%%%%%%%%
% Sets margins
%%%%%%%%%%%%%
\newgeometry{left=1.5in,right=1in,top=1in,bottom=1in}
\setlength\headsep{3pt}

%%%%%%%%%%%%%
% Creates problem and solution environments
%%%%%%%%%%%%%

% Solution Environment
\newenvironment{solution}{\begin{proof}[Solution]}{\end{proof}}

% Problem Environment
\newenvironment{problem}[1]
    {\begin{mdframed}[default]
    \textbf{Problem #1:}
    }
    {\end{mdframed}
    }
    
%%%%%%%%%%%
% Custom Commands
%%%%%%%%%%%
\newcommand{\gOne}{\cellcolor{green!50!white} 1}
\newcommand{\rZero}{\cellcolor{red!50!white} 0}

\begin{document}

\begin{problem}{\S 2.3: 2}
Determine whether $f$ is a function from $\mathbb{Z}$ to $\mathbb{R}$ if
\begin{enumerate}
    \item[(a)] $f(n) = \pm n$
    \item[(b)] $f(n) = \sqrt{n^2+1}$
    \item[(c)] $f(n) = \frac{1}{n^2-4}$
\end{enumerate}
\end{problem}

\begin{problem}{\S 2.3: 12}
Determine whether each of these functions from $\mathbb{Z}$ to $\mathbb{Z}$ is one-to-one.
\begin{enumerate}
    \item[(a)] $f(n) = n-1$.
    \item[(b)] $f(n) = n^2+1$.
    \item[(c)] $f(n) = n^3$.
    \item[(d)] $f(n) = \lceil n/2 \rceil$.
\end{enumerate}
\end{problem}

\begin{problem}{\S 2.3: 14(a,b,c,d)}
Determine whether $f: \mathbb{Z} \times \mathbb{Z} \rightarrow \mathbb{Z}$ is onto if
\begin{enumerate}
    \item[(a)] $f(m,n) = 2m-n$.
    \item[(b)] $f(m,n) = m^2 - n^2$.
    \item[(c)] $f(m,n) = m+n+1$.
    \item[(d)] $f(m,n) = |m| - |n|$.
\end{enumerate}
\end{problem}

\begin{problem}{\S 2.3: 20}
Give an example of a function from $\mathbb{N}$ to $\mathbb{N}$ that is
\begin{enumerate}
    \item[(a)] one-to-one but not onto.
    \item[(b)] onto but not one-to-one.
    \item[(c)] both onto and one-to-one (but not the identity function).
    \item[(d)] neither one-to-one nor onto.
\end{enumerate}
\end{problem}

\begin{problem}{\S 2.3: 22(a,b)}
Determine whether each of these functions is a bijection from $\mathbb{R}$ to $\mathbb{R}$.
\begin{enumerate}
    \item[(a)] $f(x) = -3x + 4$.
    \item[(b)] $f(x) = -3x^2 + 7$.
\end{enumerate}
\end{problem}

\begin{problem}{\S 2.3: 36}
Find $f \circ g$ and $g \circ f$ where $f(x) = x^2 + 1$ and $g(x) = x+2$ are functions from $\mathbb{R}$ to $\mathbb{R}$.
\end{problem}

\begin{problem}{\S 2.3: 39}
    Show that the function $f(x) = ax + b$ from $\mathbb{R}$ to $\mathbb{R}$ is invertible, where $a$ and $b$ are constants, with $a \neq 0$, and find the inverse of $f$.
\end{problem}

\begin{problem}{\S 2.3: 40(a)}
Let $f$ be a function from the set $A$ to the set $B$. Let $S$ and $T$ be subsets of $A$. Show that $f(S \cup T) = f(S) \cup f(T)$.
\end{problem}

\begin{problem}{\S 2.3: 44(b)}
Let $f$ be a function from $A$ to $B$. Let $S$ and $T$ be subsets of $B$. Show that $f^{-1}(S \cap T) = f^{-1}(S) \cap f^{-1}(T)$.
\end{problem}

\begin{problem}{\S 3.1: 2}
Determine which characteristics of an algorithm described in the text the following procedures have and which they lack.
\begin{enumerate}
    \item[(a)]
    \begin{verbatim}
        procedure double(n: positive integer)
        while n > 0
            n := 2n
    \end{verbatim}
    \item[(b)]
    \begin{verbatim}
        procedure divide(n: positive integer)
        while n >= 0
            m : = 1/n
            n := n-1
    \end{verbatim}
    \item[(c)]
    \begin{verbatim}
        procedure sum(n: positive integer)
        sum := 0
        while i < 10
            sum := sum + i
    \end{verbatim}
    \item[(d)]
    \begin{verbatim}
        procedure choose(a,b: integers)
        x := either a or b
    \end{verbatim}
\end{enumerate}
\end{problem}

\begin{problem}{\S 3.1: 24}
Describe an algorithm that determines whether a function from a finite set to another finite set is one-to-one.
\end{problem}

\begin{problem}{\S 3.1: 52(a,d)}
Use the greedy algorithm to make change using quarters, dimes, nickels, and pennies for
\begin{enumerate}
    \item[(a)] 87 cents.
    \item[(d)] 33 cents.
\end{enumerate}
\end{problem}

\begin{problem}{\S 3.1: 54(a,d)}
Use the greedy algorithm to make change using quarters, dimes, and pennies (but no nickels) for
\begin{enumerate}
    \item[(a)] 87 cents.
    \item[(d)] 33 cents.
\end{enumerate}
\end{problem}

\end{document}